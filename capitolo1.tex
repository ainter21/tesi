\chapter{Computational STEAM}
\label{chap:steam}
Il termine STEM indica un percorso di studi incentrato su quattro discipline: Scienze, Matematica, Tecnologia e Ingegneria \cite{stem}. L’acronimo nasce nel 2001 negli Stati Uniti per raggruppare le principali aree di studio necessarie per lo sviluppo e l’innovazione nell’era dell’informazione. La sua evoluzione in STEAM vuole unire le quattro discipline sopra citate all’Arte, in modo da proporre un piano volto alla formazione di ragazzi in grado sia di applicare il metodo scientifico per la risoluzione di problemi più o meno complessi, sia di sviluppare la propria fantasia e il pensiero critico. Ai giorni nostri, infatti, non basta essere fortemente competenti in ambito scientifico e tecnologico, ma è necessario anche essere creativi, originali e innovativi. STEAM si pone quindi l’obiettivo di completare quello che STEM proponeva anni prima, integrandolo ed adattandolo per essere più efficace e aggiornato. 

L’informatica, però, non è espressa direttamente in STEAM, ma è inglobata nel concetto di Technology. Questo può rivelarsi riduttivo se si associa questa disciplina all’aspetto meramente tecnico, in quanto essa si occupa dello studio di metodi generali per risolvere problemi mediante sistemi di calcolo, ed utilizza il calcolatore solamente come strumento \cite{informatica} per accelerare il processo computazionale.

L’aspetto pratico e tecnico è sicuramente più visibile ai “non addetti ai lavori”, ma non è l’unico: l’informatica può essere vista ad un livello più alto, come una scienza che si occupa di problem solving (risoluzione di problemi). L’informatico quindi non deve solo analizzare i problemi, ma deve essere capace anche di proporre una soluzione corretta ed efficiente, raggiungibile grazie a precisione, creatività e ragionamento. "Pensare come un informatico è molto piú che saper programmare un computer. Significa saper ragionare su piú livelli di astrazione" \cite{wing}.

L’informatica, inoltre, assume ulteriore importanza come supporto per lo sviluppo delle altre materie scientifiche. Ha rivoluzionato lo studio e l’analisi dei dati raccolti da altre discipline semplificando e abbreviando enormemente i tempi di calcolo, tanto da diventare fondamentale nei laboratori di ricerca. La scienza computazionale (è così chiamata qualsiasi ramo delle scienze che utilizza la potenza di calcolo offerta dai moderni calcolatori per risolvere problemi \cite{computational_science}) è una disciplina che si sta sviluppando enormemente, tanto da diventare il terzo “pilastro” dell’indagine scientifica, affiancando teoria e sperimentazione \cite{ensuring_america_competitivness}.

Da queste considerazioni nasce l’idea di un percorso di approfondimento chiamato Computational STEAM proposto dal Dipartimetno di Ingegneria e Scienza dell'Informazione di Trento (DISI), mirato a rafforzare le competenze di informatica all’interno del Liceo Scientifico, opzione Scienze Applicate. Si propone di integrare questa disciplina con le altre materie scientifiche insegnate, offrendo agli studenti gli strumenti necessari per poter approfondire e toccare con mano come queste discipline si stanno sviluppando nel mondo del lavoro e della ricerca, senza però tralasciare le materie umanistiche e artistiche, indispensabili per sviluppare un pensiero critico.

L'aggiunta della parola Computational vuole indicare inoltre che il percorso di studi si appoggia su quello che Wing nel 2006 chiama Pensiero Computazionale \cite{wing}, ovvero "un'abilità fondamentale, non solo per informatici che riguarda la risoluzione di problemi, il design di sistemi. Include una serie di strumenti che riguardano l'ambito dell'informatica.". 

L'idea quindi consiste nel formare studenti in ambito scientifico con adeguate conoscenze e competenze, ma allo stesso tempo con capacità di risolvere problemi piú o meno complessi e di prendere decisioni attraverso un pensiero critico \cite{ct_to_stem}.


\section{La situazione in Italia}
\label{sec:italia}
L’idea di adattare gli studi superiori scientifici ai principi del modello STEAM si appoggia alla riforma Gelmini del 2010, più precisamente la sezione dedicata al Liceo Scientifico con l’aggiunta del piano chiamato Scienze Applicate. Questo percorso di studi guida lo studente ad approfondire e a sviluppare le conoscenze e le abilità necessarie per seguire lo sviluppo della ricerca scientifica e tecnologica \cite{scienze_applicate}. Vengono rafforzate e approfondite discipline quali tecnologia, matematica, fisica, chimica, biologia, scienze della terra e informatica. Con particolare riferimento a quest’ultima, nel Decreto Ministeriale 211 del 7 ottobre 2010 “Indicazioni Nazionali”, allegato F \cite{riforma} si fa riferimento alla necessità di “utilizzare gli strumenti per la risoluzione di problemi appresi per la soluzione di problemi significativi connessi principalmente allo studio delle altre discipline, acquisendo la consapevolezza dei vantaggi e dei limiti dell’uso degli strumenti e dei metodi informatici e delle conseguenze sociali e culturali di tale uso”. Alla fine di tale percorso lo studente è in grado di utilizzare i principali software, scegliendo di volta in volta il più adatto.

Il collegamento con le altre materie scientifiche e umanistiche è già presente su carta nel decreto, ma ha avuto difficoltà a svilupparsi nella pratica per varie ragioni: in primo luogo, il numero di ore offerte non sono sufficienti per poter sviluppare in maniera esaustiva gli argomenti, dando la possibilità agli studenti di poter applicare gli strumenti appresi in un contesto interdisciplinare, mantenendo comunque un adeguato equilibrio tra teoria e pratica. Inoltre la mancanza di docenti tecnici da affiancare durante le esercitazioni \cite{problemi_riforma} e l’adeguamento dei testi all’utilizzo di nuove tecnologie, più comprensibili e intuitive per i ragazzi, hanno contribuito a questo rallentamento. 
Le basi di questo percorso di studi partono da questa riforma, che però non dà molta libertà, per una mancanza oggettiva di tempo in primis.

L’Italia, durante il XIX e il XX secolo, è stata una tra le prime nazioni al mondo per innovazione e progresso. Il XXI secolo, che ci ha catapultati nell’era dell’informazione, necessita di un ulteriore sforzo per rimanere al passo. Con questa proposta si vuole dare ulteriore spazio all’informatica ed alla scienza computazionale all’interno degli istituti superiori italiani, in modo da poter offrire un percorso di studi adatto a coloro che vogliono sviluppare competenze trasversali in ambito scientifico, sfruttando la potenza degli odierni calcolatori, mantenendo un giusto equilibrio con le materie umanistiche indispensabile per lo sviluppo del pensiero critico.

\section{Temi ed argomenti affrontati}

Come detto già in precedenza, questo percorso formativo avrà una forte componente trasversale e interdisciplinare: l’informatica deve essere insegnata insieme e come supporto con altre discipline scientifiche, trasmettendo l’importanza del suo ruolo nella società moderna.
Le tematiche affrontate non sono inserite in percorsi “verticali” di specializzazione, ma di più ampio respiro in modo da offrire le basi, seguendo lo spirito generalista del liceo rimandando gli approfondimenti agli studi universitari.

\textbf{Computational science} L’utilizzo dell’informatica come strumento per lo studio di modelli matematici/informatici è alla base della ricerca moderna, come anche l’utilizzo di questi modelli applicati a fenomeni reali, raccogliendo i dati e iterando il processo più volte, seguendo i principi di design ingegneristico.

\textbf{Data Science} La crescente digitalizzazione ha portato ad un esponenziale accumulo di dati raccolti. Questi dati, per poter essere utili, devono essere elaborati attraverso adeguati modelli statistici e matematici. Il ruolo del data scientist sta diventando sempre più importante nella società odierna ed è essenziale capire come i nostri dati vengono elaborati in modo da migliorare la nostra esperienza in rete. Esistono numerosi dati disponibili liberamente che possono essere utilizzati per esperienze laboratoriali, cosicché gli studenti possano capire quali sono le basi su cui si appoggiano i numerosi algoritmi utilizzati dalle grandi aziende.

\textbf{Intelligenza artificiale} Oggigiorno siamo circondati da sistemi “intelligenti” che cercano di semplificare la nostra vita. Numerose scelte vengono prese da algoritmi di cui non è chiesto conoscerne l’implementazione. È indispensabile per uno studente del liceo scientifico comprendere i principali, in modo da diventare un cittadino e non un semplice consumatore passivo.  In questo senso, il ruolo delle discipline umanistiche, storiche e filosofiche è fondamentale.


