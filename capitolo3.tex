\newpage
\chapter{Relazione tirocinio}
\label{cha:conclusioni}


In questo capitolo viene esposto un resoconto del tirocinio avvenuto tra i mesi di aprile e maggio 2019 presso il Liceo Galileo Galilei (Trento), il liceo Leonardo Da Vinci (Trento) e l’istituto tecnico tecnologico Buonarroti-Pozzo (Trento). L’attività è stata strutturata in due parti: una parte teorica di lezione interattiva e una parte laboratoriale in cui i ragazzi svolgevano degli esercizi. Lo scopo del tirocinio è stato quello di proporre un modo diverso di visualizzare ed utilizzare l’informatica presso gli istituti superiori, seguendo i principi del modello Computational STEAM (\autoref{chap:steam}). L’argomento trattato, i protocolli epidemici (\autoref{chap:epidemic}), è stato scelto perché adatto a far comprendere come modelli e algoritmi studiati in altri campi (la biologia e la matematica) possano essere adattati ed utilizzati con efficacia in ambito informatico (nello specifico la comunicazione in sistemi distribuiti).
\section{Contesto e organizzazione del corso}
Essendo tre studenti abbiamo organizzato questa esperienza di tironcio nel seguente modo: ognuno ha portato un argomento diverso, ma con le stesse modalità di presentazione (seguendo i principi di Computatonal STEAM) in tre classi per un totale di 9 classi.
Abbiamo presentato il corso tra aprile e maggio presso i seguenti istituti superiori: liceo Scientifico Galileo Galilei di Trento, liceo Scientifico Leonardo Da Vinci di Trento e istituto tecnico e tecnologico Buonarroti-Pozzo. Abbiamo scelto questo periodo in quanto più adatto sia per i ragazzi che per gli insegnanti ordinari. Ci siamo concentrati sul liceo scientifico, in quanto il percorso di studi Computational STEAM è indirizzato a questa tipologia di scuola, ma abbiamo deciso di provare a portare l’esperienza anche in un istituto tecnico, per cercare di capire quali possono essere le differenze riguardo la preparazione dei ragazzi, in modo da adattare le lezioni al meglio. Abbiamo strutturato le lezioni nel seguente modo: un’ora per l’introduzione, e le restanti 6 ore per il contenuto del corso in sè. La lezione introduttiva e l’ultima sono state utilizzate anche per compilare un questionario introduttivo ed un conclusivo (entrambi anonimi), i cui risultati verranno utilizzati come supporto alla stesura di questo capitolo. È stata utilizzata l’applicazione Moduli Google. Abbiamo raccolto le risposte di 164 ragazzi per il questionario iniziale, 161 per quello finale. Questi sono i dati aggregati per tutte le classi. Utilizziamo i dati aggregati per la parte  di contesto, mentre per la parte "tecnica" li suddividiamo in base al corso frequentato (alcune domande sono specifiche del modulo, come quelle riguardante il feedback riscontrato, il quale può variare in base a chi ha presentato il corso).
Le classi a cui sono rivolti sono terze, quarte e quinte, perché gli argomenti trattati risultano più adatti a studenti con conoscenze che vengono fornite a partire dal terzo anno in poi. La scelta delle classi è stata fatta dagli insegnanti di informatica, a cui sono stati proposti i moduli.
\begin{figure}[!h]
    \begin{subfigure}{.5\textwidth}
        \centering
        \includegraphics[width=\textwidth]{genere.png}
        \caption{Genere}
        \label{fig:genere}
    \end{subfigure}\hfill
    \begin{subfigure}{.5\textwidth}
        \centering
        \includegraphics[width=\textwidth]{classi.png}
        \caption{Classi}
        \label{fig:classi}
    \end{subfigure}
    \caption{Caratteristiche del campione} 
\end{figure}
\subsection{Analisi del campione}
Le classi sono prevalente composte da maschi, il che non è affatto sorprendente osservando la composizione nei corsi di studio di informatica ed ingegneria. I ragazzi hanno notato questa differenza numerica, motivandola principalmente con questa affermazione: “l'informatica è preferita dai ragazzi”. Nonostante ciò, la figura dell’informatico è stata descritta in modo abbastanza lontano dai frequenti stereotipi: secondo i dati raccolti molti pensano che un informatico debba essere creativo e debba avere capacità nel lavorare in gruppo. 

Ai ragazzi è stato chiesto quanto apprezzano la materia in sè e come questa venga insegnata attualmente: i risultati presentano un giusto equilibrio, con alcuni studenti che la preferiscono ed altri che la disprezzano o non ne trovano l'utilità. Dai dati raccolti si può notare infatti che piú della metà del campione non ha un'idea ben chiara degli utilizzi dell'informatica, specialmente in altri ambiti quali la biologia, la fisica, la matematica e l'arte. Generalmente infatti l'informatica come scienza computazionale non è approfondita negli istituti superiori, nonostante il corso di studi lo preveda \cite{riforma}. Le lezioni sono comunque organizzate mantenendo un giusto rapporto tra teoria e pratica.

\section{Contenuto e metodo esecutivo} 
Il contenuto del corso è stato esposto utilizzando delle slide per la parte di lezione teorica, mentre per la parte applicativa è stata fornita una cartella condivisa con gli esercizi da svolgere e le loro spiegazioni nel dettaglio (\href{http://www.tinyurl.com/protocolli-epidemici}{http://www.tinyurl.com/protocolli-epidemici}). La parte teorica è stata presentata come lezione interattiva, dando spazio agli studenti di porci delle domande e di rispondere a quesiti da noi proposti. Ad eccezione della prima ora di lezione, tutte le altre hanno avuto una componente pratica, in modo tale che i ragazzi potessero applicare quanto imparato. La difficoltà degli esercizi è via via crescente, sia per la complessità degli algoritmi proposti, sia per un minore aiuto fornito in determinate situazioni (in alcuni casi è stato fornito solamente lo pseudocodice). 
Abbiamo scelto di sviluppare il corso sui protocolli epidemici (vedi Capitolo 2) per diversi motivi: in primo luogo, l’argomento si basa su semplici regole che vengono utilizzate più volte in diversi contesti. I protocolli di gossip possono però essere studiati maggiormente e lasciano spazio anche ad un approfondimento autonomo. Inoltre sono sono un argomento interdisciplinare: vengono utilizzati in reti distribuite per la comunicazione, ma i principi su cui si basano sono studiati in epidemiologia, nell’ambito quindi della diffusione delle epidemie, ma anche dalle scienze sociali, riguardo il fenomeno del gossip. Questa interdisciplinarietà si dimostra adatta all’interno di questa esperienza che vuole portare i principi di Computational STEAM nelle scuole superiori.
Nello specifico ho sviluppato il programma del corso nel seguente modo: nella prima lezione, utile come introduzione, ho presentato l’argomento e la sua natura interdisciplinare, il contesto storico, la definizione del problema, le applicazioni reali e i principi di base.
\begin{figure}[!h]
    
    \begin{subfigure}{.5\textwidth}
        \centering
        \includegraphics[width=\textwidth]{pull.png}
        \captionsetup{justification=centering}
        \caption{Algoritmo modello SI: pull} 
    \end{subfigure}\hfill
    \begin{subfigure}{.5\textwidth}
        \centering
        \includegraphics[width=\textwidth]{feedback-counter.png}
        \captionsetup{justification=centering}
        \caption{Algoritmo modello SIR: feedback-counter} 
    \end{subfigure}
    \caption{Esempi di output}
    \label{fig:output}
\end{figure} 
Nelle successive lezioni mi sono concentrato, dopo aver fatto una breve introduzione sui grafi (utile per alcune definizioni ed assunzioni), sui due modelli SI e SIR presentando gli algoritmi e stili che li utilizzano, lasciando poi spazio ai ragazzi di implementarli su schermo. L’ambiente di sviluppo utilizzato è Processing (\href{https://processing.org}{https://processing.org}), libreria grafica di Java e IDE utile per creare simulazioni e metodi di visualizzazione efficace. Gli studenti si sono occupati di scrivere gli algoritmi, completando il codice fornito oppure interpretando lo pseudo codice (l'output di alcuni esercizi in Figura \ref{fig:output}). Ho fornito una libreria (\href{https://github.com/ainter21/epidemic}{https://github.com/ainter21/epidemic}), utile sia per alleggerire il carico di lavoro, sia per nascondere alcune componenti troppo complesse, come la parte di visualizzazione (lo scopo degli esercizi è quello di implementare l’algoritmo, mentre la parte di visualizzazione è solamente un supporto per rendere più chiari i concetti).
Ho fornito la documentazione utile per completare gli esercizi, con la spiegazione dei metodi e degli attributi. Il loro obiettivo è stato quindi utilizzare la libreria fornita (con metodi molto simili a quelli utilizzati nell' pseudocodice), affinché gli algoritmi proposti funzionassero. La lista degli esercizi proposti già completati è scaricabile qui (\href{https://github.com/ainter21/epidemic-protocols}{https://github.com/ainter21/epidemic-protocols}).
Abbiamo svolto le attività nei laboratori di informatica, in modo tale da poter proiettare i lucidi e permettere ai ragazzi di lavorare in autonomia o a gruppi. L'idea iniziale era di formare dei gruppi di lavoro per poter collaborare e sviluppare abilità di problem solving e lavoro di squadra. Questa soluzione si è rivelata di difficile attuazione in quanto i ragazzi possiedono un account personale non condivisibile e se alcuni studenti rimanevano assenti, si rischiava di non poter accedere al lavoro iniziato la volta precedente. Abbiamo comunque fatto il possibile per invogliarli a lavorare in gruppo.

\subsection{Libreria nel dettaglio}
La libreria è stata scritta in modo tale da nascondere la parte di programmazione ad oggetti non studiata dai ragazzi al liceo scientifico. Le classi principali sono: \texttt{Graph}, \texttt{Node} e \texttt{Info}. La rete è visualizzata attraverso un grafo. Abbiamo associato il valore interno del nodo ad un colore, per rendere l'output visivo piú accattivante.

I ragazzi hanno utilizzato i metodi già implementati nei vari oggetti (specialmente nella classe \texttt{Node}) per eseguire azioni e acquisire informazioni. Il grafo creato è completo, assunzione che durante le lezioni non è stata rilassata per mancanza di tempo.

I ragazzi hanno comunque avuto la possibilità di leggere il codice e di modificarlo, compilando la propria versione della libreria.
\section{Analisi dei risultati}

\begin{figure}[h]
    
    \begin{subfigure}{.5\textwidth}
        \centering
        \includegraphics[width=\textwidth]{aspettative.png}
        \captionsetup{justification=centering}
        \caption{Ritieni che siano state \\ soddisfatte le tue aspettative?} 
    \end{subfigure}\hfill
    \begin{subfigure}{.5\textwidth}
        \centering
        \includegraphics[width=\textwidth]{conoscenze.png}
        \captionsetup{justification=centering}
        \caption{Quanto il corso era adatto  \\ al tuo livello di conoscenze?} 
    \end{subfigure}
    \caption{Risultati}
\end{figure} 

Dopo le sette ore previste, abbiamo chiesto di compilare un questionario di valutazione delle attività svolte.  I dati analizzati in questa sezione non sono i dati aggregati: 60 risposte per il questionario iniziale, e 52 per il questionario finale, ovvero quelle riguardanti il modulo "protocolli epidemici".
I risultati sono positivi, la maggior parte degli studenti dice di essere rimasto soddisfatto. Gli argomenti trattati erano una novità per la maggior parte dei ragazzi, ed alcuni hanno riferito di voler approfondire l'argomento (sia per quanto riguarda l'utilizzo di Processing, sia per i protocolli epidemici, sia per la diffusione delle epidemie). 

Avendo portato il corso sia in un liceo che in un istituto tecnico abbiamo notato alcune differenze, a causa del programma svolto dai ragazzi e dal numero di ore assegnate all'informatica. I ragazzi del liceo, non avendo programmato ad oggetti, si sono trovati in difficoltà nel comprendere alcuni concetti. Conoscevamo già il programma svolto ed il livello di preparazione, ma abbiamo comunque optato per l'utilizzo di Processing e quindi di Java, in quanto avrebbe reso piú accattivante il risultato finale per i ragazzi. Alcuni ragazzi hanno avuto piú difficoltà nel lavorare in gruppo perché, da quanto capito, non è una tecnica molto utilizzata durante l'anno. Inoltre si sono trovati spaesati nel momento in cui abbiamo dato loro autonomia nello svolgere gli esercizi: alcuni hanno avuto difficoltà nel completare gli esercizi. Ai ragazzi è stato suggerito di scrivere su carta, riflettere in gruppo per cercare una possibile soluzione, ma questo consiglio non è stato molto seguito e la maggior parte ha iniziato a scrivere immediatamente codice, con risultati non sempre soddisfacienti. In alcuni casi è stato mostrato dello pseudocodice in modo da dare un punto di partenza: all'istituto tecnico era già stato utilizzato questo metodo, quindi questo materiale è risultato piú utile.

Anche l'utilizzo di una libreria e' stata una novita' per i ragazzi: hanno sperimentato questa potenzialita' dei linguaggi di programmazione consultando la documentazione fornita, cercando di capire quali metodi e attributi fossero adatti alla situazione. Anche se all'inizio si sono trovati un po' spaesati, con il procedere del corso hanno preso dimestichezza con il metodo proposto.

Sarebbe stato anche utile investire alcune ore per introdurre in modo piú approfondito l'ambiente di sviluppo, ma purtroppo il tempo era limitato e questo avrebbe ridotto lo spazio dedicato all'argomento principale del corso. Il codice Processing può essere scritto in diversi linguaggi di programmazione oltre a Java: Python e Javascript. Si potrebbe provare a portare questi corsi utilizzando il linguaggio Python, piú semplice rispetto a Java (è stato scelto Java perché simile al C++, utilizzato durante l'anno dai ragazzi). Python presenta alcuni vantaggi rispetto ad altri linguaggi, specialmente nel contesto degli istituti superiori: come viene fatto notare da Grandell e colleghi in \cite{python_high_school}, questo linguaggio ha una sintassi piú semplice rispetto ai linguaggi normalmente utilizzati, è tipizzato dinamicamente (può essere uno svantaggio, ma riduce la quantità di codice) ed oggigiorno è sempre piú utilizzato anche nel mondo del lavoro. Inoltre le potenzialità di Python emergono anche dal grande numero di moduli che possono essere aggiunti. 

In conclusione si può dire che l'esperienza ha avuto esito positivo leggendo i feedback dei ragazzi ed è stato utile a noi per capire in che modo proporre un argomento esterno al programma scolastico presentandolo con modalità diverse da quelle classiche. è stato utile ai ragazzi per capire come l'informatica possa essere applicata ed utilizzata come materia interdisciplinare e di supporto e come altre materie si possono interfacciare all'informatica ("ho conosciuto delle sue applicazioni molto interessanti ", "Ho cambiato idea sull'applicazione dell'informatica in campi diversi, penso che sia un buon supporto!"). Nel caso venga riproposta un'esperienza simile, è auspicabile che i ragazzi abbiano già un conoscenza di base riguardo la programmazione ad oggetti e l'ambiente di sviluppo Processing in modo da rendere piú comprensibili alcuni concetti e dare piú libertà ai ragazzi nello scrivere il codice (il dover partire da zero a scrivere il codice leggendo la documentazione della libreria è stato uno scoglio troppo grande da superare, quindi gli esercizi sono stati presentati in modo guidato, cone delle parti da completare).
