\chapter*{Conclusioni} % senza numerazione
\label{conclusioni}

\addcontentsline{toc}{chapter}{Conclusioni} % da aggiungere comunque all'indice

La proposta di un corso di studi superiori basato sui principi di STEAM e con l'integrazione dell'informatica non solo come materia per tecnici del settore, può essere un grande passo in avanti nella riforma dell'istruzione in Italia. L'informatica negli anni ha assunto sempre più importanza nelle nostre vite grazie al rapido sviluppo tecnologico: è quindi necessario formare dei ragazzi consapevoli delle potenzialità che questa disciplina può offrire, anche come supporto ad altre materie, sia scientifiche che umanistiche.

Il test eseguito durante il tirocinio presso i licei di Trento, e descritto nel \autoref{cha:tirocinio}, si tratta di un primo passo, atto a capire l'interesse per questa proposta da parte degli insegnanti ed il feedback dei ragazzi. Nonostante il tempo limitato, sono state raccolte informazioni e critiche utili per sviluppare una proposta più articolata e completa. La scelta dell'argomento, i protocolli epidemici, si è rivelata efficace in quanto perché si basa su concetti semplici, facilmente assimilabili dai ragazzi, senza però cadere nel banale o nel monotono. Il carico di lavoro assegnato ai ragazzi non è stato oneroso, in quanto gli esercizi sono sempre stati svolti in orario scolastico, e la parte teorica affrontata si è sviluppata in superficie, fornendo gli strumenti necessari per capire senza sovraccaricare la mole di informazioni. Il metodo di insegnamento è ulteriormente migliorabile, sia con l'esperienza sia con alcuni accorgimenti: una maggiore interazione con gli studenti e un'introduzione più approfondita riguardo l'ambiente di sviluppo.

I supporti utilizzati sono stati efficaci, anche se l'utilizzo di Processing ha rallentato alcuni passaggi e limitato lo svolgimento degli esercizi, in quanto per la maggior parte degli studenti era una novità, sia dal punto di vista del linguaggio, sia dal punto di vista dell paradigma di programmazione ad oggetti, nonostante la libreria fornita astraesse alcuni concetti troppo complessi per essere spiegati in poco tempo. 

L'utilizzo dei Moduli Google per la raccolta di dati, sia all'inizio che al termine, si è rivelata efficace ed indispensabile per la stesura della relazione di tirocinio, fornendo un supporto di valutazione per l'esperienza del docente. La maggior parte dei ragazzi ha risposto in modo coerente e composto, e questo ha permesso di analizzare un campione ampio e vario. 

Personalmente, ritengo che l'esperienza sia stata molto costruttiva e stimolante, perché mi ha permesso di interfacciarmi con un mondo a me nuovo, venendo a contatto il metodo di insegnamento e scontrandomi con le difficoltà che mi si sono presentate davanti durante il percorso. Ho sfruttato le mie capacità acquisite durante il percorso di studi svolto in questi tre anni all'Università di Trento in un ambiente diverso cercando di trasmettere la mia passione per questa disciplina.

