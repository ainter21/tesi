\chapter*{Sommario} % senza numerazione
\label{sommario}

\addcontentsline{toc}{chapter}{Sommario} % da aggiungere comunque all'indice

Nel 2010, grazie alla riforma Gelmini \cite{riforma}, la struttura degli istituti superiori ha subito cambiamenti ed alcuni nuovi indirizzi sono stati aggiunti: tra questi il liceo Scientifico - opzione Scienze Applicate in sostituzione del liceo Scientifico PNI. In questo istituto la legge si pone l'obiettivo di dare più importanza all'informatica, ponendo maggiore attenzione sulle potenzialità di questa materia anche come supporto per lo studio di altre materie, in ambito scientifico, ma anche umanistico. Sono passati nove anni dall'approvazione del decreto e dall'introduzione di questo indirizzo negli istituti superiori, ma alcune linee guida sono ancora di difficile attuazione specialmente per mancanza di docenti tecnici di supporto durante le esercitazioni e per il mancato adeguamento alle nuove tecnologie. 

L'acronimo STEM nasce nel 2001 negli Stati Uniti, indica un percorso di studi incentrato su Scienze, Tecnologia, Ingegneria e Matematica, che ha come obiettivo quello di integrare queste materie in un sistema di apprendimento attivo basato su un continuo confronto con applicazioni reali \cite{stem_education}. 

L'aggiunta dell'arte all'interno di questo curriculum, con il relativo cambio di nome da STEM a STEAM, si poggia su un semplice concetto: gli studenti, per avere successo nel mondo del lavoro e della ricerca, devono essere sia metodici e competenti, ma allo stesso tempo devono sfruttare la propria creatività ed il pensiero critico. 

L'ultimo passo, che porta alla nascita di Computational STEAM, viene eseguito integrando l'informatica, già presente all'interno della macro area Tecnologia, non solo come materia a sè stante, ma anche come supporto per le altre discipline. La scienza computazionale sfrutta infatti le potenzialità computazionali dell'informatica per analizzare, organizzare ed elaborare grandi quantità di dati.

Il Dipartimento di Ingegneria e Scienze dell'Informazione (DISI), propone un percorso di studio indirizzato a studenti del liceo Scientifico, chiamato Computational STEAM, con lo scopo di rafforzare le competenze di informatica, integrando questa materia con le altre discipline scientifiche, per offrire agli studenti gli strumenti necessari per capire e toccare con mano come queste ultime si stanno sviluppando nel mondo del lavoro e della ricerca, senza però tralasciare le materie umanistiche, indispensabili per sviluppare un pensiero critico.

Le tematiche affrontate non sono inserite in percorsi specializzanti, ma di più ampio respiro in modo da offrire le basi agli studenti, rimandando poi gli approfondimenti agli studi universitari (seguendo lo spirito generalista che contraddistingue i licei). Durante le ore dedicate all'informatica viene dato spazio ad argomenti quali Data Science e Intelligenza Artificiale, indispensabili oggigiorno sia nel mondo del lavoro che in quello della ricerca sia se si intende proseguire la propria formazione in facoltà di Ingegneria, ma anche in settori quali matematica, fisica, biologia, \dots

Il punto cardine del percorso di studi Computational STEAM rimane comunque quello di formare uno studente in grado di utilizzare l'informatica in ambienti fortemente interdisciplinari. Le potenzialità sono molteplici: dall'analisi di dati in scienze, matematica e fisica alla creazione di simulazione di ambienti reali quali la diffusione delle epidemie o lo studio del volo degli uccelli fino ad arrivare agli studi dell'arte orientale con i suoi pattern regolari che si poggiano su regole matematiche.

L'esperienza di tirocinio, svolta tra i mesi di aprile e maggio presso il Liceo Galileo Galilei (Trento), il liceo Leonardo Da Vinci (Trento), l’istituto tecnico tecnologico Buonarroti-Pozzo (Trento) e il liceo Scientifico presso il collegio Arcivescovile Celestino Endrici (Trento) è stato un primo test per capire se e come questo percorso di studi potesse essere introdotto ai ragazzi. 
Lo stage è stato organizzato da tre studenti ed ognuno ha scelto un argomento diverso tra quelli riportati sopra (più precisamente lo studio delle epidemie e l'utilizzo dei modelli per la comunicazione in sistemi distribuiti, un approfondimento sui frattali e l'analisi di modelli di simulazione ad agente per osservare il comportamento delle formiche ed il volo degli uccelli), seguendo i principi proposti da Computational STEAM quali apprendimento attivo, lavoro in gruppo e risoluzione di problemi (problem solving). Nello specifico questo elaborato riporta un'analisi del modulo proposto riguardante i protocolli epidemici, organizzato su sette ore e presentato in tre classi, di cui due del liceo Scientifico e una dell'istituto tecnico. L'argomento è stato scelto per diverse ragioni: in primo luogo, riesce ad esprimere l'interdisciplinarietà dell'informatica, caratteristica fondamentale per dare uno spunto di riflessione ai ragazzi sulle potenzialità della materia. Infatti, i modelli utilizzati in epidemiologia (il modello SI ed il modello SIR), sono stati adattati e rielaborati per sviluppare alcuni dei protocolli per la distribuzione delle informazioni in sistemi distribuiti. Inoltre gli algoritmi si basano su regole semplici, adatte ad essere comprese dai ragazzi dal terzo anno in poi, ma che allo stesso tempo possono essere approfondite in modo da risolvere problemi complessi. 

L'elaborato è suddiviso in tre capitoli: Computational STEAM, Protocolli epidemici nel dettaglio e Relazione di tirocinio. Nel primo vengono esposte le motivazione che hanno portato alla proposta di questo nuovo percorso di studi indirizzato ai licei, con un'analisi delle problematiche che rallentano il processo di adeguamento alla riforma ed una serie di proposte per la sua attuazione nei prossimi anni. 

Il secondo capitolo espone nel dettaglio l'argomento tecnico affrontato durante il corso, con ulteriori approfondimenti non trattati durante le ore scolastiche. Durante queste ultime infatti, sono stati presentati i due principali modelli, SI e SIR, con i relativi algoritmi e stili. Gli studenti hanno avuto la possibilità di implementarli su schermo utilizzando l'ambiente di sviluppo Processing e sfruttando una libreria con le classi necessarie per completare gli esercizi proposti. La libreria è stata fornita ad inizio corso, con la relativa documentazione da cui i ragazzi hanno potuto comprenderne il funzionamento. Il secondo capitolo espone anche ulteriori utilizzi dei protocolli epidemici rispetto la disseminazione di informazioni, quali Aggregazione, Peer Sampling e rilevamento di fallimenti all'interno della rete. L'argomento è analizzato dal punto di vista tecnico esponendo le differenze e un'analisi della complessità degli algoritmi.

Il terzo capitolo consiste in una descrizione dell'attività svolta presso gli istituti, l'organizzazione del corso ed il metodo esecutivo seguito. I ragazzi hanno compilato due questionari, creati utilizzando i Moduli Google: uno iniziale per una raccolta di pareri riguardo il loro modo di concepire l'informatica, come questa venga insegnata in classe e per capire il loro livello di conoscenze riguardo l'argomento presentato durante le lezioni successive. Il questionario finale è stato invece utilizzato per raccogliere i pareri dell'esperienza e per capire se gli studenti hanno notato alcune differenze rispetto al metodo di insegnamento "classico". è presente inoltre una descrizione della libreria, scritta in Java, utilizzata per astrarre alcuni concetti non studiati dai ragazzi e per gestire la componente di visualizzazione, necessaria per la generazione di un output accattivante e significativo. I ragazzi hanno avuto la possibilità di visualizzarla, di compilarla e di utilizzare la propria versione oppure di scaricare una versione già precompilata fornita ad inizio del corso. Infine è presente l'analisi dei risultati ottenuti durante l'esperienza di tirocinio con relativi commenti ed opinioni, per un'eventuale riproposta del modulo, o della sua integrazione all'interno di un corso di studi più ampio, come quello di Computational STEAM.

L'esperienza nel complesso è sicuramente positiva e, nonostante il tempo limitato, gli argomenti sono stati trattati in maniera esaustiva. I ragazzi sono rimasti soddisfatti e, da quanto riportato, le tematiche sono state esposte in modo chiaro. Alcuni hanno lamentato un mancato approfondimento ulteriore riguardo il linguaggio di programmazione, che per mancanza di tempo non è avvenuta. la scelta dell'utilizzo di Processing, nonostante gli studenti non l'avessero mai utilizzato, si è rivelata comunque utile dal punto di vista dell'efficacia di visualizzazione: ai ragazzi è servito per mantenere alta l'attenzione e l'interesse. Gli esercizi proposti non hanno creato molti problemi, perchè alcuni sono stati presentati in modo abbastanza guidato, in modo da prendere confidenza con l'ambiente di sviluppo. 
