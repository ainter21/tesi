\chapter*{Sommario} % senza numerazione
\label{sommario}

\addcontentsline{toc}{chapter}{Sommario} % da aggiungere comunque all'indice

Nel 2010, grazie alla riforma Gelmini \cite{riforma}, la struttura degli istituti superiori ha subito cambiamenti ed alcuni nuovi indirizzi sono stati aggiunti: tra questi il liceo Scientifico - opzione Scienze Applicate in sostituzione del liceo Scientifico PNI. In questo istituto la legge si pone l'obiettivo di dare piu' importanza all'informatica, ponendo maggiorne attenzione sulle potenzialita' di questa materia anche come supporto per lo studio di altre materie, in ambito scientifico, ma anche umanistico. Sono passati nove anni dall'approvazione del decreto e dall'introduzione di questo indirizzo negli istituti superiori, ma alcune linee guida sono ancora di difficile attuazione specialmente per mancanza di docenti tecnici di supporto durante le esercitazioni e per il mancato adeguamento alle nuove tecnologie. 

Il Dipartimento di Ingengeria e Scienze dell'Informazione (DISI), propone un percorso di studio indirizzato a studenti del liceo Scientifico, chiamato Computational STEAM, con lo scopo di rafforzare le competenze di informatica, integrando questa materia con le altre discipline scientifiche, per offrire agli studenti gli strumenti necessari per capire e toccare con mano come queste ultime si stanno sviluppando nel mondo del lavoro e della ricerca, senza pero' tralasciare le materie umanistiche, indispensabili per sviluppare un pensiero critico.

L'acronimo STEM nasce nel 2001 negli Stati Uniti, indica un percorso di studi incentrato sulle seguente discipline: Scienze, Tecnologia, Ingegneria e Matematica (Science, Technology, Engineering and Mathematics). Aggiungendo l'Arte 

Il sommario dell’elaborato consiste al massimo di 3 pagine e deve contenere le seguenti informazioni:
\begin{itemize}
  \item contesto e motivazioni 
  \item breve riassunto del problema affrontato
  \item tecniche utilizzate e/o sviluppate
  \item risultati raggiunti, sottolineando il contributo personale del laureando/a
\end{itemize}




